%
% File acl2017.tex
%
%% Based on the style files for ACL-2015, with some improvements
%%  taken from the NAACL-2016 style
%% Based on the style files for ACL-2014, which were, in turn,
%% based on ACL-2013, ACL-2012, ACL-2011, ACL-2010, ACL-IJCNLP-2009,
%% EACL-2009, IJCNLP-2008...
%% Based on the style files for EACL 2006 by 
%%e.agirre@ehu.es or Sergi.Balari@uab.es
%% and that of ACL 08 by Joakim Nivre and Noah Smith

\documentclass[11pt,a4paper]{article}
\usepackage[hyperref]{acl2017}
\usepackage{times}
\usepackage{latexsym}

\usepackage{url}

\aclfinalcopy % Uncomment this line for the final submission
%\def\aclpaperid{***} %  Enter the acl Paper ID here

%\setlength\titlebox{5cm}
% You can expand the titlebox if you need extra space
% to show all the authors. Please do not make the titlebox
% smaller than 5cm (the original size); we will check this
% in the camera-ready version and ask you to change it back.

\newcommand\BibTeX{B{\sc ib}\TeX}

\title{Computational characterization of mental states:  A natural language processing approach}


\author{Facundo Carrillo \\
  Computer Science Dept, School of Science, \\
  Buenos Aires University \\
  {\tt fcarrillo@dc.uba.ar} }
\date{}

\begin{document}
\maketitle
\begin{abstract}
 Psychiatry is an area of medicine that strongly bases its diagnoses on the psychiatrist’s subjective appreciation. The task of diagnosis loosely resembles the common pipelines used in supervised learning schema. Therefore, we propose to augment the psychiatrists’ diagnosis toolbox with an artificial intelligence system based on natural language processing and machine learning algorithms. This approach has been validated in many works in which the performance of the  diagnosis has been increased with the use of automatic classification. 
 
 \end{abstract}
 

\section{Introduction}

Psychiatry is an area of medicine that strongly bases its diagnoses on the psychiatrist’s subjective appreciation. More precisely, speech is used almost exclusively as a window to the patient’s mind. Few other cues are available to objectively justify a diagnostic, unlike what happens in other disciplines which count on laboratory tests or imaging procedures, such as X-rays. Daily practice is based on the use of semi-structured interviews and standardized tests to build the diagnoses, heavily relying on her personal experience. This methodology has a big problem: diagnoses are commonly validated a posteriori in function of how the pharmacological treatment works. This validation cannot be done until months after the start of the treatment and, if the patient condition does not improve, the psychiatrist often changes the diagnosis and along with the pharmacological treatment. This delay prolongs the patient's suffering until the correct diagnosis is found. According to NIMH, more than 1\% and 2 \% of US population is affected by Schizophrenia and Bipolar Disorder, respectively. Moreover, the WHO reported that the global cost of mental illness reached \$2.5T in 2010 \cite{mathers2008global} .

The task of diagnosis, largely simplified, mainly consists of understanding the mind state through the  extraction of patterns from the patient's speech and finding the best matching pathology in the standard diagnostic literature. This pipeline, consisting of extracting patterns and then classifying them, loosely resembles the common pipelines used in supervised learning schema. Therefore, we propose to augment the psychiatrists’ diagnosis toolbox with an artificial intelligence system based on natural language processing and machine learning algorithms. The proposed system would assist in the diagnostic using a patient’s speech as input. The understanding and insights obtained from customizing these systems to specific pathologies is likely to be more broadly applicable to other NLP tasks, therefore we expect to make contributions not only for psychiatry but also within the computer science community. We intend to develop these ideas and evaluate them beyond the lab setting. Our end goal is to make it possible for a practitioner to integrate our tools into her daily practice with minimal effort.


\section{Methodology}
In order to complement the manual diagnosis it is necessary to have samples from real patients. To collect these samples, we have ongoing collaborations with different psychiatric institutions from many countries: United States, Colombia, Brazil and Argentina. These centers provide us with access to the relevant patient data and we jointly collaborate testing different protocols in a variety languages. We have already started studies with two pathologies: Schizophrenia and Bipolar Disorder.

Regarding our technical setup, we are using and developing tools to capture different characteristics of the speech. In all cases, we work with high-quality transcriptions of speech. Our experiments are focused on analyzing different aspects of the speech: 1) Grammatical-morphological changes based on topology of Speech Graphs. 2) Coherence Algorithm: Semantic coherence using proximity in semantic embeddings. 3) Changes in Emotional language and other semantic categories

\section{Preliminary Results}
Many groups have already validated this paradigm \cite{roark2011spoken,fraser2014automated,resnik2013using,lehr2012fully,fraser2016linguistic,mitchell2015quantifying}. First, Speech Graphs has been used in different pathologies (schizophrenic and bipolar), results are published in \cite{carrillo2014automated,mota2014graph,mota2012speech}. In \cite{carrillo2014automated}, the autors can automatically diagnose based on the graphs with an accuracy greater than 85\%. This approach consists in modeling the language, or a transformation of it (for example the part of speech symbols of a text), as a graph. With this new representation the authors use graph topology features (average grade of nodes, number of loops, centrality, etc) as features of patient speech. 
Regarding coherence analysis, some researchers has developed an algorithm that quantifies the semantic divergence of the speech. To do that, they used semantic embeddings (like Latent Semantic Analysis\cite{landauer1997solution}, Word2vec\cite{mikolov2013distributed}, or Twitter Semantic Similarity\cite{carrillo2015fast}) to measure when consecutive sentences of spontaneous speech differ too much. The authors used this algorithm, combined with machine learning classifiers, to predict which high-risk subjects would have their first psychotic episode within 2 years (with 100\% accuracy) \cite{bedi2015automated}. The latter result was very relevant because it presented evidence that this automatic diagnostic methodology could not only perform at levels comparable to experts’ but also, under some conditions, even outperform experts (classical medical tests achieved 40\% of accuracy). Dr. Insel, former director of National Institute of Mental Health  cited this work in his blog on his post: Look who is getting into mental health research as one.

Regarding Emotional language, some researchers presented evidence of how some endocrine regulations change the language. This methods are based on quantifying different emotional levels. This methodology has been used to diagnose depression and postpartum depression by Eric Horvitz \cite{de2013predicting}. Others researchers also have used this method to diagnose patients with Parkinson’s disease\cite{garcia2016language}.


\section{Current Work}
Currently, we are working on the coherence algorithm \cite{bedi2015automated}, understanding some properties and its potential applications, such as automatic composition of text and feature extraction for bot detection. Meanwhile, we are receiving new speech samples from 3 different mental health hospitals in Argentina provided by patients with new pathologies like frontotemporal dementia and anxiety. We are also building methods to detect depression in young patients using the change of emotions in time.


\section{Future work}
The tasks for the following 2/3 years are:  1) Improve implementations of developed algorithms and make them open source. 2) Integrate the different pipelines of features extraction and classification to generate a generic classifier for several pathologies. 3) Build a mobile application for medical use (for this aim, Google has awarded our project with the Google Research Awards for Latin America 2016: “Prognosis in a Box: Computational Characterization of Mental State”). At the moment the data is recorded and then transcribed by an external doctor. We want a full automatic procedure, from the moment when the doctor performs the interview to the moment when she receives the results. 4) Write the PhD thesis.

 

% 
% In an effort to accommodate the color-blind (as well as those printing
% to paper), grayscale readability for all accepted papers will be
% encouraged.  Color is not forbidden, but authors should ensure that
% tables and figures do not rely solely on color to convey critical
% distinctions.
% Here we give a simple criterion on your colored figures, if your paper has to be printed in black and white, then you must assure that every curves or points in your figures can be still clearly distinguished.

% Min: no longer used as of ACL 2017, following ACL exec's decision to
% remove this extra workflow that was not executed much.
% BEGIN: remove
%% \section{XML conversion and supported \LaTeX\ packages}

%% Following ACL 2014 we will also we will attempt to automatically convert 
%% your \LaTeX\ source files to publish papers in machine-readable 
%% XML with semantic markup in the ACL Anthology, in addition to the 
%% traditional PDF format.  This will allow us to create, over the next 
%% few years, a growing corpus of scientific text for our own future research, 
%% and picks up on recent initiatives on converting ACL papers from earlier 
%% years to XML. 

%% We encourage you to submit a ZIP file of your \LaTeX\ sources along
%% with the camera-ready version of your paper. We will then convert them
%% to XML automatically, using the LaTeXML tool
%% (\url{http://dlmf.nist.gov/LaTeXML}). LaTeXML has \emph{bindings} for
%% a number of \LaTeX\ packages, including the ACL 2017 stylefile. These
%% bindings allow LaTeXML to render the commands from these packages
%% correctly in XML. For best results, we encourage you to use the
%% packages that are officially supported by LaTeXML, listed at
%% \url{http://dlmf.nist.gov/LaTeXML/manual/included.bindings}
% END: remove

% \section{Translation of non-English Terms}
% 
% It is also advised to supplement non-English characters and terms
% with appropriate transliterations and/or translations
% since not all readers understand all such characters and terms.
% Inline transliteration or translation can be represented in
% the order of: original-form transliteration ``translation''.

% \section{Length of Submission}
% \label{sec:length}
% 
% The ACL 2017 main conference accepts submissions of long papers and
% short papers.
%  Long papers may consist of up to eight (8) pages of
% content plus unlimited pages for references. Upon acceptance, final
% versions of long papers will be given one additional page -- up to nine (9)
% pages of content plus unlimited pages for references -- so that reviewers' comments
% can be taken into account. Short papers may consist of up to four (4)
% pages of content, plus unlimited pages for references. Upon
% acceptance, short papers will be given five (5) pages in the
% proceedings and unlimited pages for references. 
% 
% For both long and short papers, all illustrations and tables that are part
% of the main text must be accommodated within these page limits, observing
% the formatting instructions given in the present document. Supplementary
% material in the form of appendices does not count towards the page limit.
% 
% However, note that supplementary material should be supplementary
% (rather than central) to the paper, and that reviewers may ignore
% supplementary material when reviewing the paper (see Appendix
% \ref{sec:supplemental}). Papers that do not conform to the specified
% length and formatting requirements are subject to be rejected without
% review.
% 
% Workshop chairs may have different rules for allowed length and
% whether supplemental material is welcome. As always, the respective
% call for papers is the authoritative source.
% 
% \section*{Acknowledgments}
% 
% The acknowledgments should go immediately before the references.  Do
% not number the acknowledgments section. Do not include this section
% when submitting your paper for review.

% include your own bib file like this:
% \bibliographystyle{acl}
% \bibliography{acl2017}
\bibliography{acl2017}
\bibliographystyle{acl_natbib}
% 
% \appendix
% 
% \section{Supplemental Material}
% \label{sec:supplemental}
% ACL 2017 also encourages the submission of supplementary material
% to report preprocessing decisions, model parameters, and other details
% necessary for the replication of the experiments reported in the 
% paper. Seemingly small preprocessing decisions can sometimes make
% a large difference in performance, so it is crucial to record such
% decisions to precisely characterize state-of-the-art methods.
% 
% Nonetheless, supplementary material should be supplementary (rather
% than central) to the paper. {\bf Submissions that misuse the supplementary 
% material may be rejected without review.}
% Essentially, supplementary material may include explanations or details
% of proofs or derivations that do not fit into the paper, lists of
% features or feature templates, sample inputs and outputs for a system,
% pseudo-code or source code, and data. (Source code and data should
% be separate uploads, rather than part of the paper).
% 
% The paper should not rely on the supplementary material: while the paper
% may refer to and cite the supplementary material and the supplementary material will be available to the
% reviewers, they will not be asked to review the
% supplementary material.
% 
% Appendices ({\em i.e.} supplementary material in the form of proofs, tables,
% or pseudo-code) should come after the references, as shown here. Use
% \verb|\appendix| before any appendix section to switch the section
% numbering over to letters.
% 
% \section{Multiple Appendices}
% \dots can be gotten by using more than one section. We hope you won't
% need that.

\end{document}
