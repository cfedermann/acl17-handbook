\documentclass[11pt]{article}
\usepackage[utf8]{inputenc} 
\usepackage[T1]{fontenc} % fonts to encode unicode
\usepackage{times}
\sloppy
\hyphenpenalty 10000

% for Letter size

%\setlength\topmargin{0.2cm} \setlength\oddsidemargin{-0cm}
%\setlength\textheight{22cm} \setlength\textwidth{15.8cm}
%\setlength\columnsep{0.25in}  \newlength\titlebox \setlength\titlebox{2.00in}
%\setlength\headheight{5pt}   \setlength\headsep{0pt}
%\setlength\footskip{1.0cm}
%\setlength\leftmargin{0.0in}
%\pagestyle{empty}
%%%%%%%%%%%%%%%%%%%%%%%%%%%%%%%%%%%%%%%%%%%%%%%%%%%%%%%%%%%%%%%%


% for A4 size

\setlength\topmargin{-5mm} %\setlength\oddsidemargin{-0cm}
\setlength\oddsidemargin{0.3cm}
\setlength\textheight{24.7cm} \setlength\textwidth{16cm}
\setlength\columnsep{0.6cm}  \newlength\titlebox \setlength\titlebox{2.00in}
\setlength\headheight{5pt}   \setlength\headsep{0pt}
\setlength\footskip{1.0cm}
\setlength\leftmargin{0.0in}
\pagestyle{empty}
%%%%%%%%%%%%%%%%%%%%%%%%%%%%%%%%%%%%%%%%%%%%%%%%%%%%%%%%%%%%%%%%


\setlength{\parindent}{0in}
\setlength{\parskip}{2ex}

\begin{document}

\begin{center}
{\Large Invited Talk \hfill\\}
\vspace{3mm}
  {\LARGE \bf  Rational Distortions of \hfill\\}
{\LARGE \bf Learners' Linguistic Input \hfill\\}
\vspace{6mm}
{\Large \bf  Naomi Feldman - University of Maryland}\hfill\\

\end{center}

\vspace*{0.5cm}


%%%%%%%%%%%%%%%%%%%%%%%%%%%%%%%%%%%%%%%%%%%%%%%%%%%%%%%%%%%%%%%%%%%%%%%%

%%% INSERT YOUR INTRO HERE



\paragraph{Abstract:}
Language acquisition can be modeled as a statistical inference problem: children use sentences and sounds in their input to infer linguistic structure.  However, in many cases, children learn from data whose statistical structure is distorted relative to the language they are learning.  Such distortions can arise either in the input itself, or as a result of children's immature strategies for encoding their input.  This work examines several cases in which the statistical structure of children's input differs from the language being learned.  Analyses show that these distortions of the input can be accounted for with a statistical learning framework by carefully considering the inference problems that learners solve during language acquisition
\paragraph{Bio:} Naomi Feldman is an associate professor in the Department of Linguistics and the Institute for Advanced Computer Studies at the University of Maryland.  She received her PhD in Cognitive Science from Brown University in 2011.   Her research lies at the intersection of cognitive science, computer science, and linguistics.  She uses methods from machine learning to create formal models of how people learn and represent the structure of their language, and has been developing methods that take advantage of naturalistic speech corpora to study how listeners encode information from their linguistic environment.


\end{document}