\documentclass[11pt]{article}
\usepackage[utf8]{inputenc} 
\usepackage[T1]{fontenc} % fonts to encode unicode
\usepackage{times}
\sloppy
\hyphenpenalty 10000

% for Letter size

%\setlength\topmargin{0.2cm} \setlength\oddsidemargin{-0cm}
%\setlength\textheight{22cm} \setlength\textwidth{15.8cm}
%\setlength\columnsep{0.25in}  \newlength\titlebox \setlength\titlebox{2.00in}
%\setlength\headheight{5pt}   \setlength\headsep{0pt}
%\setlength\footskip{1.0cm}
%\setlength\leftmargin{0.0in}
%\pagestyle{empty}
%%%%%%%%%%%%%%%%%%%%%%%%%%%%%%%%%%%%%%%%%%%%%%%%%%%%%%%%%%%%%%%%


% for A4 size

\setlength\topmargin{-5mm} %\setlength\oddsidemargin{-0cm}
\setlength\oddsidemargin{0.3cm}
\setlength\textheight{24.7cm} \setlength\textwidth{16cm}
\setlength\columnsep{0.6cm}  \newlength\titlebox \setlength\titlebox{2.00in}
\setlength\headheight{5pt}   \setlength\headsep{0pt}
\setlength\footskip{1.0cm}
\setlength\leftmargin{0.0in}
\pagestyle{empty}
%%%%%%%%%%%%%%%%%%%%%%%%%%%%%%%%%%%%%%%%%%%%%%%%%%%%%%%%%%%%%%%%


\setlength{\parindent}{0in}
\setlength{\parskip}{2ex}

\begin{document}

\begin{center}
{\Large Invited Talk \hfill\\}
\vspace{3mm}
  {\LARGE \bf  Should Neural Network Architecture \hfill\\}
{\LARGE \bf Reflect Linguistic Structure? \hfill\\}
\vspace{6mm}
{\Large \bf  Chris Dyer - DeepMind/CMU}\hfill\\

\end{center}

\vspace*{0.5cm}


%%%%%%%%%%%%%%%%%%%%%%%%%%%%%%%%%%%%%%%%%%%%%%%%%%%%%%%%%%%%%%%%%%%%%%%%

%%% INSERT YOUR INTRO HERE



\paragraph{Abstract:}
I explore the hypothesis that conventional neural network models (e.g., recurrent neural networks) are incorrectly biased for making linguistically sensible generalizations when learning, and that a better class of models is based on architectures that reflect hierarchical structures for which considerable behavioral evidence exists. I focus on the problem of modeling and representing the meanings of sentences. On the generation front, I introduce recurrent neural network grammars (RNNGs), a joint, generative model of phrase-structure trees and sentences. RNNGs operate via a recursive syntactic process reminiscent of probabilistic context-free grammar generation, but decisions are parameterized using RNNs that condition on the entire (top-down, left-to-right) syntactic derivation history, thus relaxing context-free independence assumptions, while retaining a bias toward explaining decisions via "syntactically local" conditioning contexts. Experiments show that RNNGs obtain better results in generating language than models that don’t exploit linguistic structure. On the representation front, I explore unsupervised learning of syntactic structures based on distant semantic supervision using a reinforcement-learning algorithm. The learner seeks a syntactic structure that provides a compositional architecture that produces a good representation for a downstream semantic task. Although the inferred structures are quite different from traditional syntactic analyses, the performance on the downstream tasks surpasses that of systems that use sequential RNNs and tree-structured RNNs based on treebank dependencies. This is joint work with Adhi Kuncoro, Dani Yogatama, Miguel Ballesteros, Phil Blunsom, Ed Grefenstette, Wang Ling, and Noah A. Smith.

\paragraph{Bio:} Chris Dyer is a research scientist at DeepMind and an assistant professor in the School of Computer Science at Carnegie Mellon University. In 2017, he received the Presidential Early Career Award for Scientists and Engineers (PECASE). His work has occasionally been nominated for best paper awards in prestigious NLP venues and has, much more occasionally, won them. He lives in London and, in his spare time, plays cello.

\end{document}