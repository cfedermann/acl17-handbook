\documentclass[11pt,a4paper]{article}
\usepackage[nohyperref]{acl2017}
\usepackage{amsmath}
\usepackage{amssymb}
\usepackage[pdftex]{graphicx}
\usepackage{times}
\usepackage{latexsym}
\usepackage{url}

\aclfinalcopy

\DeclareMathOperator*{\argmax}{arg\,max}

\title{Oracle Summaries of Compressive Summarization}

\author{
Tsutomu Hirao \and Masaaki Nishino \and Masaaki Nagata\\
NTT Communication Science Laboratories, NTT Corporation\\
2-4 Hikaridai, Seika-cho, Soraku-gun, Kyoto, 619-0237, Japan\\
{\tt \{hirao.tsutomu,nishino.masaaki,nagata.masaaki\}@lab.ntt.co.jp }
}


\date{}
\pagestyle{empty}

\begin{document}
\maketitle
\begin{abstract}
This paper derives an Integer Linear Programming (ILP) formulation to
obtain an oracle summary of the {\it compressive summarization} paradigm in
terms of {\sc Rouge}.
The oracle summary is essential to reveal the upper bound performance of
the paradigm.
Experimental results on the DUC dataset showed that {\sc Rouge} scores
of compressive oracles are significantly higher than those of extractive
oracles and state-of-the-art summarization systems.
These results reveal that compressive summarization is a promising
 paradigm and encourage us to continue with the research to produce
 informative summaries.
\end{abstract}

\section{Introduction}

{\it Compressive summarization}, a joint model
integrating sentence extraction and sentence compression within a unified framework, 
has been attracting attention in recent years \citep{martins-smith09,kirkpatrick11,Almeida13,Qian13,kikuchi:2014,Yao:IJCAI15}.
Since compressive summarization methods can use a sub-sentence as an
atomic unit, they can pack more information into summaries
than extractive methods, which employ sentences as atomic units.
Thus, compressive summarization is essential when we want to produce
summaries under tight length constraints.
There are two approaches to compress entire document(s) to be
grammatical; one is trimming the phrase structure trees
\citep{kirkpatrick11} and the other is trimming the dependency trees
obtained from the document(s)
\citep{martins-smith09,Almeida13,Qian13,kikuchi:2014,Yao:IJCAI15}. This paper focuses on
 the latter approach because recently it has been receiving much attention.

To measure the performance of compressive summarization methods,
{\sc Rouge} \citep{rouge}, an automatic evaluation metric, is widely
used. {\sc Rouge} evaluates a system summary by exploiting a set of
human-made reference summaries to give a score in the range [0,1]. When
n-gram occurrences of the system summary agree with those in a set of
reference summaries, the value is 1. However, system summaries cannot
achieve {\sc Rouge}${=}1$ since summarization systems cannot reproduce
reference summaries in most cases. In other words, the maximum {\sc Rouge}
score that can be achieved by compressive summarization is unclear. As a
result, researchers cannot know how much room for further improvement is left.
Thus, it is beneficial to reveal the upper bound summary that achieves the maximum {\sc
Rouge} score and can be produced by the systems.
The upper bound summary is known as the oracle summary.
To obtain the oracle summary on extractive summarization paradigms,
several approaches have been proposed.
\citet{sipos-shivaswamy-joachims:2012:EACL2012} utilized a greedy algorithm, and
\citet{kubina-conroy-schlesinger:2013:MultiLing} utilized exhaustive
search based on heuristics. However, their oracle summaries do not always
retain the optimal (maximum) {\sc Rouge} score.
Recently, \citet{hirao:eacl2017} derived an Integer Linear Programming (ILP)
formulation to obtain the optimal oracle summary.
Their oracle summary can help researchers to 
comprehend the strict limitation of the extractive
summarization paradigm.
However, their method cannot be applied to obtain compressive oracle
summaries.

To reveal the ultimate limitation of the compressive summarization
paradigm, we propose an ILP formulation to
obtain a compressive oracle summary that maximizes the {\sc Rouge} score.
We conducted experimental evaluation on the Document Understanding Conference
(DUC) 2004 dataset. The result
demonstrated that {\sc Rouge} scores of compressive oracle summaries completely
outperformed those of extractive oracle summaries and those
of state-of-the-art summarization methods.
This indicates that compressive summarization is a promising paradigm for leveraging research resources.


\section{Definition of Compressive Oracle Summaries}

Before defining compressive oracle summary, we briefly describe
{\sc Rouge}$_n$. Given $K$ reference summaries
$\boldsymbol{R}{=}\{R_1,\ldots,R_K\}$ and a system summary $S$. Let $G{=}\{g_1^n,\ldots,g_M^n\}$
be the set of all n-grams appearing in reference summaries. Let $|G|{=}M$.
{\sc Rouge}$_n$ is defined as
follows:
\begin{equation}
\label{rouge}
\begin{split}
\text{\sc Rouge}_n(\boldsymbol{R},S)=&\\
&\kern-7em \frac{\sum_{k=1}^{K} \sum_{j=1}^{M} \min\{N(g_j^n,
 R_k), N(g_j^n, S)\} }{\sum_{k=1}^K\sum_{j=1}^M N(g_j^n, R_k) }
\end{split}
\end{equation}
$g_j^n$ represents the $j$-th n-gram appearing in reference summaries.
$N(g_j^n,R_k)$ and $N(g_j^n,S)$ are the number of occurrences of 
n-gram $g_j^n$ in $R_k$ and $S$, respectively.
Thus, compressive oracle summaries are defined as follows:
\begin{equation}
\label{def:oracle}
\begin{split}
O=&\argmax_{S \subseteq T} \text{\sc Rouge}_n(\boldsymbol{R},S)\\
& {\rm s.t.} ~~~\ell(S) \le L_{\rm max}.
\end{split}
\end{equation}
$T$ is the set of all valid word subsequences\footnote{
Word subsequences can be regarded as grammatical sentences.
We regard rooted subtrees of dependency trees as 
valid word subsequences.
For details, see Section 3.1.
} obtained
from sentences contained in the input document(s), and
$L_{\rm max}$
is the length limitation of the oracle summary.
 $\ell(S)$ indicates
the number of words in the summary.
Neither approximation nor exact algorithms are known for solving this problem.

\begin{figure*}[tb]
 \begin{center}
  \scalebox{0.7}[0.6]{\includegraphics{exampledoc_small_rev3.pdf}}
    \caption{Examples of trees that represent dependency relations
  between chunks, and word sequences (whose length is 2).
  Chunks are enclosed in square
    brackets.  
  Note that we disregard word sequences that are generated by destroying
  the structure of chunks such as ``live\_every'' in $S_1$, ``dolphins\_in'' in $S_2$, 
  ``live\_wild'' in $S_3$.
}
  \label{deptree}
 \end{center}
\end{figure*}

\section{ILP Formulation to Obtain the Compressive Oracle Summary}

\subsection{Dependency Structure of a Sentence}

In this paper, we follow the dependency tree trimming approach proposed by 
Filippova et al. \citeyearpar{filippova08,filippova:2013}.
They proposed rules that transform a tree that represents dependency relation between
words into a tree that represents dependency relation between chunks 
(consisting of a word or word sequence). 
Since we can trim their dependency trees without loss of grammatical consistency,
Thus, we employ the trees in our compressive summarization
framework.
Figure \ref{deptree} shows examples.

 
\subsection{ILP Formulation}

\begin{eqnarray}
   \text{maximize}
  & \displaystyle\sum_{k=1}^{K}\sum_{j=1}^{M} z_{k,j}\\
 \text{s.t.} & \displaystyle\sum_{i=1}^{|D|} \sum_{u=1}^{E_i}\ell_{i,u} b_{i,u} \le L_{\rm max}\\
\forall j :      & \displaystyle\sum_{\tau \in {\cal T}(g_j^n)} m_{\tau} \ge z_{k,j}\\
\forall j :      &  N(g_j^n,R_k) \ge z_{k,j}\\
\forall i,u :    &  b_{i,{\rm parent}(i,u)} \ge b_{i,u}\\
\forall i, v, q \in V_i(w_{i,v}) : &  b_{i,q} \ge m_{i,v}\\
\forall i, v, p \in U_i(w_{i,v}) : &  b_{i,p} \le 1- m_{i,v}\\
\displaystyle\forall i,u :   &  m_{i,v} \in \{0,1\}\\
\displaystyle\forall i,v :   &  b_{i,u} \in \{0,1\}\\
\displaystyle\forall k,j :   &  z_{k,j} \in \mathbb{Z}_{+}.
\end{eqnarray}

\noindent
Since the denominator of equation (1) is constant
for a given set of reference summaries, we
can find an oracle summary by maximizing the numerator
of equation (1).
Equation (3) is the objective function that corresponds to maximization of
the numerator of equation (1). 
$z_{k,j}$ is the count of the $j$-th n-gram that is contained in both 
the $k$-th reference summary and the oracle summary.
Equation (4) ensures that the length of the oracle summary is less than
$L_{\rm max}$.
$b_{i,u}$ is a binary decision variable indicating whether $u$-th chunk in $i$-th
sentence is contained in an oracle summary or not. $\ell_{i,u}$ indicates the number
of the words in $u$-th chunk in the $i$-th sentence.
$D$ is a set of sentences and $E_i$ is the number of chunks in the $i$-th sentence.
Equations (5) and (6) represent $\min$ operation in equation (1).
$w_{i,v}$
 is the $v$-th possible word sequence whose length is $n$ and that is contained in the $i$-th sentence, and $m_{i,v}$ is a binary decision variable indicating whether
$w_{i,v}$ 
is contained in the oracle summary or not.
${\cal T}(g_j^n)$ is a set of tuples consisting of indices $(i,v)$ whose
word sequence
corresponds to $g_j^n$, {\it i.e.},
${\cal T}(g_j^n){=}\{(i,v)|w_{i,v}{=}g_j^n\}$.
Thus, $z_{k,j}{=}\min\{N(g_j^n,R_k),N(g_j^n,S)\}$.
Equation (7) ensures that an oracle summary consists of a set of
rooted subtrees of the sentences in the entire document(s).
Function parent ($i,u$) returns the index of the parent chunk of the $u$-th
chunk in the dependency tree obtained from the $i$-th sentence.
Equations (8) and (9) represent the dependency relation between n-grams and
chunks.
When we 
include $w_{i,v}$ in the oracle summary,
we have to include all chunks that contain the words in $w_{i,v}$.
In addition, when the above chunks have gap(s), we have to drop chunk(s) within
the gap(s).
Here, $V_i(w_{i,v})$ is a set of indices of
chunks that includes words in $w_{i,v}$, and $U_i(w_{i,v})$ is a set of
indices of chunks within the gap(s), defined as 
 $\{h| \min(V_i(w_{i,v}))<h< \max(V_i(w_{i,v}))\}$ and 
 $h \notin
 V_i(w_{i,v})$.

We give an example to show how chunks and word sequences are related.
When we pack a bigram ``live\_in'' in an oracle summary, there are
four candidates in the source document (Fig. \ref{deptree}).
Word subsequences, $w_{1,6}$,$w_{2,5}$,$w_{2,6}$ and $w_{3.9}$ match ``live\_in''.
Thus, 
 ${\cal T}(\text{live\_in})=\{(1,6),(2,5),(2,6),(3,9)\}$.
Here, when we want to pack $w_{2,6}$ into the oracle summary,
we have to pack both chunks $c_{2,2}$ and $c_{2,4}$ ($b_{2,2}=b_{2,4}=1$) because
$U_2(w_{2,6})=\{2,4\}$. Then, we have to drop chunk $c_{2,3} (b_{2,3}=0)$ because 
$c_{2,3}$ is within the gap between chunks $c_{2,2}$ and $c_{2,4}$ ($V_2(w_{2,6})=3$).
Similarly, when we pack $w_{3,9}$ into an oracle summary, we have to
pack both chunks $c_{3,3}$ and $c_{3,5}$ and drop chunk $c_{3,4}$.
However, this compression is not allowed since 
 there is no dependency relationship between $c_{3,3}$ and $c_{3,5}$.

After solving the ILP problem, we can obtain compressive oracle summaries
by collecting chunks according to $b_{i,u}{=}1$.

\section{Experiments}

To investigate the potential limitation of the compressive summarization
paradigm, we compare {\sc Rouge} scores of compressive oracle summaries
with those of extractive oracle summaries and those obtained from state-of-the-art
summarization systems.
Extractive oracle summaries
are obtained by solving the ILP formulation proposed by \citep{hirao:eacl2017}.
System summaries are extracted from 
 a public
 repository\footnote{
\url{http://www.cis.upenn.edu/~nlp/corpora/sumrepo.html}} \cite{HONG14LREC}.

\subsection{Settings}

We conducted experimental evaluation on the DUC-2004 dataset for multiple document
summarization evaluation, a widely used benchmark test set for generic multiple
document summarization tasks. The dataset consists of 50 topics, each
of which contains 10 newspaper articles.
To obtain oracle summaries based on the ILP formulation described in section 3.2,
first, we applied the Stanford parser \citep{stanford06} to all
sentences in the dataset
to obtain dependency relations between words, and then we transformed them into trees
that represent the dependency relations between chunks by applying
 Filippova's rules \citep{filippova08,filippova:2013}.
To solve the ILP problem, we utilized {\tt CPLEX version 12.5.1.0}.

We obtained and evaluated oracle summaries based on three variants of {\sc Rouge},
$\text{\sc Rouge}_{1}$, $\text{\sc Rouge}_{2}$ and $\text{\sc
Rouge-SU0}$, with the following conditions\footnote{
With stop words: options ``-n 2 -s -m -x'' are used.
Without stop words: options ``-n 2 -m -x'' are used.
}:
(1) $\text{\sc Rouge}_{1}$, utilizing unigrams excluding stopwords
(2) $\text{\sc Rouge}_{2}$, utilizing bigrams with stopwords, and
(3) $\text{\sc Rouge-SU0}$, which is an extension of {\sc Rouge}$_n$,
utilizing unigram and bigram (excluding skip-bigram) statistics.


\subsection{Results and Discussion}

\begin{table}[tb]
 \begin{center}
  {\tabcolsep=0.95mm
  \begin{tabular}{ll|llll}
   \multicolumn{2}{r|}{\raisebox{-1.8ex}[0pt][0pt]{Method}} & \multicolumn{4}{c}{Metric}\\
                              &                            &  $n{=}1$ & $n{=}2$ & $n{=}1{+}2$ & Sent.\\ 
	   \hline
         & $n{=}1$ & {\bf 42.6} & 13.1 & 24.1 & 5.34\\
   Ext.  & $n{=}2$ & 36.6 & {\bf 16.9} & 24.3 & 5.06\\
         & $n{=}1{+}2$ & 40.9 & 16.1 & {\bf 25.4} & 5.24\\
   \hline
         & $n{=}1$ & {\bf 50.9} & 13.8 & 27.7 & 10.6\\
   Comp. & $n{=}2$ & 40.9 & {\bf 21.3} & 28.6 & 7.82\\
         & $n{=}1{+}2$ & 47.9 & 19.7 & {\bf 30.3} & 8.48\\
   \hline
   \multicolumn{2}{r|}{RegSum} & 33.1 & 10.2 & 18.8 & 4.9\\
   \multicolumn{2}{r|}{ICSISumm} & 31.0 & 10.3 & 18.0 & 4.2\\
  \end{tabular}
  }
  \caption{{\sc Rouge} scores and the number of sentences of extractive and compressive oracle
  summaries and those obtained from state-of-the-art summarization
  systems, RegSum   and ICSISumm. $n{=1}$
  corresponds to {\sc Rouge}$_1$, $n{=}2$ corresponds to
  {\sc Rouge}$_2$, $n{=}1{+}2$ corresponds to {\sc Rouge}-SU0. ``Sent.''
  indicates the average number of sentences in the summaries.}
  \label{oracle_rouge}
 \end{center}
\end{table}


Table \ref{oracle_rouge} shows {\sc Rouge} scores of compressive and 
extractive oracle summaries and those of RegSum
\citep{hong-nenkova:2014:EACL} that achieved the best {\sc Rouge}$_1$ and ICSISumm
\citep{gillick-favre:2009:ILPNLP,gillick:tac:09} that achieved the best 
{\sc Rouge}$_2$ on the DUC-2004 dataset, respectively. 


We compare {\sc Rouge} scores of compressive oracle summaries with
 extractive oracle summaries. 
 The best scores are obtained when we use the same {\sc Rouge} variant
 for both computation and evaluation
(see bolded scores in Table \ref{oracle_rouge}).
There are large differences between the best scores of extractive method and compressive method. 
The differences are 8.3 points, 4.4 points and 4.9 points for {\sc
 Rouge}$_1$, {\sc Rouge}$_2$, {\sc Rouge}-SU0, respectively.
As one of the reasons for the above results, 
compressive oracle summaries have a much
 larger number of (sub-)sentences than extractive oracle summaries for the
 same length limitation.
This is an advantage of compressive summarization over extractive
summarization.

However, we have to note that compressive oracle summaries optimized
to {\sc Rouge}$_1$ may not be desirable since they are produced by
compressing sentences by
ignoring contexts.
In fact, they obtained remarkable gain for {\sc Rouge}$_1$ score (8.3 points),
while they obtained modest gains in $\text{\sc Rouge}_2$ and {\sc
Rouge}-SU0 (0.7 and 3.6 points, respectively).
This may suggest that the resultant summaries overfit to the unigrams in
the reference summaries.

We compare {\sc Rouge} scores of compressive oracle summaries with
 those of system summaries, {\sc Rouge} scores of compressive oracle
 summaries completely outperformed those of state-of-the-art systems.
The differences are in a range from 11 to 17 points. 

The results demonstrated that compressive summarization is a promising 
approach to produce more informative summaries, 
and room still exists for further improvement.
Thus, compressive summarization is important research topic to leverage
our resources.
\begin{table}[tb]
 \begin{center}
  \begin{tabular}{ll|l}
    \multicolumn{2}{l|}{Method} & Score\\
   \hline
    \raisebox{-1.8ex}[0pt][0pt]{Ext.} & $n{=}1$ & 4.55\\
                                     & $n{=}2$ & 4.58 \\
   \hline
    \raisebox{-1.8ex}[0pt][0pt]{Comp.} & $n{=}1$ & 3.88\\
                                     & $n{=}2$ & 4.07 \\
  \end{tabular}
  \caption{Readability evaluation by human subjects}
  \label{subjective_eval}
 \end{center}
\end{table}

\begin{figure*}[tb]
 \begin{flushleft}
  \begin{small}
  \hrule
{\bf Reference:}\\
The Wye River accord has not been implemented.
As the Israeli cabinet was considering the agreement, Islamic Jihad militants exploded a car bomb in nearby Mahane Yehuda market.
The cabinet suspended ratification of the agreement, demanding the Palestinian Authority take steps against terrorism.
Further, after the bombing, Israeli Prime Minister Netanyahu announced the resumption of construction of a new settlement, Har Homa, in a traditionally Arab area east of Jerusalem.
Israel also demands that Arafat outlaw the military wings of Islamic Jihad and Hamas.
The attack injured 24 Israelis, but only the two assailants, Sughayer and Tahayneh, were killed.\\
\vspace{2mm}
\noindent {\bf Extractive oracle summary $n=1$:}\\
The procedure is part of the Wye River agreement negotiated last month.
The radical group Islamic Jihad claimed responsibility Saturday for the market bombing and vowed more attacks to try to block the new peace accord.
Most recently, Israel's Cabinet put off a vote to ratify the accord after a suicide bombing Friday in Jerusalem that killed the two assailants and injured 21 Israelis.
David Bar-Illan, a top aide to Israeli Prime Minister Benjamin Netanyahu, said Sunday that Israel expects Palestinian leader Yasser Arafat to formally outlaw the military wings of Islamic Jihad and the larger militant group Hamas.\\
  \vspace{2mm}
  \noindent {\bf Compressive oracle summary $n=1$:}\\
The Israeli cabinet suspended ratification of the Wye agreement.
A Prime Minister Benjamin Netanyahu said that Israel would continue to build Jewish neighborhoods throughout Jerusalem including at a site in the Arab sector of the city.
Netanyahu's Cabinet delayed action on the peace accord.
The radical group Islamic Jihad claimed responsibility for the bombing and vowed attacks.
Implementation of the Israeli-Palestinian land-for-security accord was to have begun.
David Bar-Illan said that Israel expects Palestinian Yasser Arafat to outlaw the military wings of Islamic Jihad and the Hamas.
Their car-bomb blew in a Jerusalem market killing men and wounding 24 people.\\
  \vspace{2mm}
 \noindent {\bf Extractive oracle summary $n=2$:}\\
In response to the attack, the Israeli cabinet suspended ratification of the Wye agreement until there `` is verification that the Palestinian authority is indeed fighting terrorism.''
The radical group Islamic Jihad claimed responsibility Saturday for the market bombing and vowed more attacks to try to block the new peace accord.
Most recently, Israel's Cabinet put off a vote to ratify the accord after a suicide bombing Friday in Jerusalem that killed the two assailants and injured 21 Israelis.
Their car-bomb blew apart two hours later in a Jerusalem market, killing
   both men and wounding 24 people. I'm going to Paradise. ''\\
   \vspace{2mm}
\noindent {\bf Compressive oracle summary $n=2$:}\\
The Israeli cabinet suspended ratification of the agreement.
Hassan Asfour said the Palestinian Authority condemned the attack.
Two people were killed.
The procedure is part of the Wye River agreement.
The radical group Islamic Jihad claimed responsibility for the bombing
   and vowed more attacks. Israel is demanding that the military wings of two radical Islamic groups be outlawed.
Implementation of the land-for-security accord was to have begun.
Israel's Cabinet put off a vote to ratify the accord after a bombing in Jerusalem that killed the two assailants and injured 21 Israelis.
Their car-bomb blew in a Jerusalem market killing men.\\
\hrule
\end{small}
  \caption{Summaries obtained from topic:D30010}
    \label{sum_examp}
\end{flushleft}
\end{figure*}

\subsection{Readability evaluation}

We conducted human evaluation to compare readability of extractive oracle summaries to
that of compressive oracle summaries.
We presented the oracle summaries to five human subjects and asked them
to rate the summaries using an integer scale from 1 (very poor) to 5 (very good).
Table \ref{subjective_eval} shows the results.
Extractive oracle summaries achieved near perfect scores.
Although the scores of compressive oracle summaries are inferior to those of
extractive oracle summaries, they achieved good enough score,
around 4.
 The results support that our trimming approach based on chunk
is effective. We show examples of oracle summaries 
in Figure \ref{sum_examp}.

\section{Conclusion}

To reveal the ultimate limitations of the compressive
summarization paradigm, this paper proposed an Integer Linear
Programming (ILP) formulation to obtain compressive oracle summaries in
terms of {\sc Rouge}.
Evaluation results obtained from the DUC 2004 dataset demonstrated that 
{\sc Rouge} scores of compressive summaries are significantly superior
to those of extractive oracle summaries and those of the
state-of-the-art systems. 
These results imply that the compressive summarization paradigm is a
promising direction to produce informative summaries and 
encourage leveraging of further resources for the research.
 
\bibliographystyle{acl_natbib}
\bibliography{mybib_short_euc}

\end{document}
