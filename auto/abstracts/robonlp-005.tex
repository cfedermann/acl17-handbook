Service robots are expected to operate in specific environments, where the presence of humans plays a key role. A major feature of such robotics platforms is thus the ability to react to spoken commands. This requires the understanding of the user utterance with an accuracy able to trigger the robot reaction. Such correct interpretation of linguistic exchanges depends on physical, cognitive and language-dependent aspects related to the environment. In this work, we present the empirical evaluation of an adaptive Spoken Language Understanding chain for robotic commands, that explicitly depends on the operational environment during both the learning and recognition stages. The effectiveness of such a context-sensitive command interpretation is tested against an extension of an already existing corpus of commands, that introduced explicit perceptual knowledge: this enabled deeper measures proving that more accurate disambiguation capabilities can be actually obtained.
