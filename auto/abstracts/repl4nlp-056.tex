Language in social media is a dynamic system, constantly evolving and adapting, with words and concepts rapidly emerging, disappearing, and changing their meaning. These changes can be estimated using word representations in context, over time and across locations. A number of methods have been proposed to track these spatiotemporal changes but no general method exists to evaluate the quality of these representations. Previous work largely focused on qualitative evaluation, which we improve by proposing a set of visualizations that highlight changes in text representation over both space and time. We demonstrate usefulness of novel spatiotemporal representations to explore and characterize specific aspects of the corpus of tweets collected from European countries over a two-week period centered around the terrorist attacks in Brussels in March 2016. In addition, we quantitatively evaluate spatiotemporal representations by feeding them into a downstream classification task -- event type prediction. Thus, our work is the first to provide both intrinsic (qualitative) and extrinsic (quantitative) evaluation of text representations for spatiotemporal trends.
