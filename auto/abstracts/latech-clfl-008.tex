The oral component of medieval poetry was integral to its performance and reception. Yet many believe that the medieval voice has been forever lost, and any attempts at rediscovering it are doomed to failure due to scribal practices, manuscript mouvance, and linguistic normalization in editing practices. This paper offers a method to abstract from this noise and better understand relative differences in phonological soundscapes by considering syllable qualities. The presented syllabification method and soundscape analysis offer themselves as cross-disciplinary tools for low-resource languages. As a case study, we examine medieval German lyric and argue that the heavily debated lyrical ‘I' follows a unique trajectory through soundscapes, shedding light on the performance and practice of these poets.
