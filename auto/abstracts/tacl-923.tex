Much like sentences are composed of words, words themselves are composed of smaller units. For example, the English word questionably can be analyzed as question+able+ly. However, this structural decomposition of the word does not directly give us a semantic representation of the word's meaning. Since morphology obeys the principle of compositionality, the semantics of the word can be systematically derived from the meaning of its parts. In this work, we propose a novel probabilistic model of word formation that captures both the analysis of a word w into its constituents segments and the synthesis of the meaning of w from the meanings of those segments. Our model jointly learns to segment words into morphemes and compose distributional semantic vectors of those morphemes. We experiment with the model on English CELEX data and German DerivBase (Zeller et al., 2013) data. We show that jointly modeling semantics increases both segmentation accuracy and morpheme F1 by between 3\% and 5\%. Additionally, we investigate different models of vector composition, showing that recurrent neural networks yield an improvement over simple additive models. Finally, we study the degree to which the representations correspond to a linguist's notion of morphological productivity.