Research into emoji in textual communication has, thus far, focused on high-frequency usages and the ambiguity of interpretations. Investigation of emoji uses across a wide range of uses can divide them into different linguistic functions: function and content words, or multimodal affective markers. Identifying where an emoji is merely replacing part of the text allows NLP tools the possibility of parsing them as any other word or phrase. Smiling emoticons are usually left out of data sets, but if they are used as the noun ``smile'' or the verb ``smiling'', we should be able to predict their part of speech. We report on an annotation task on English Twitter data with the goal of classifying emoji usage by these categories, and on the effectiveness of a classifier trained on these annotations. We find that it is possible to train a classifier to tell the difference between those emoji used as linguistic content words and those used as paralinguistic or affective multimodal markers even with a small amount of training data, but that accurate sub-classification of these multimodal emoji into specific classes like attitude, topic, or gesture will require more data and more feature engineering.
