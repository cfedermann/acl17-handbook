Relation extraction is the task of recognizing and extracting relations between entities or concepts in texts. A common approach is to exploit existing knowledge to learn linguistic patterns expressing the target relation and use these patterns for extracting new relation mentions. Deriving relation patterns automatically usually results in large numbers of candidates, which need to be filtered to derive a subset of patterns that reliably extract correct relation mentions. We address the pattern selection task by exploiting the knowledge represented by entailment graphs, which capture semantic relationships holding among the learned pattern candidates. This is motivated by the fact that a pattern may not express the target relation explicitly, but still be useful for extracting instances for which the relation holds, because its meaning entails the meaning of the target relation. We evaluate the usage of both automatically generated and gold-standard entailment graphs in a relation extraction scenario and present favorable experimental results, exhibiting the benefits of structuring and selecting patterns based on entailment graphs.
