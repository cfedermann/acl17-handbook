Analogy completion via vector arithmetic has become a common means of demonstrating the compositionality of word embeddings. Previous work have shown that this strategy works more reliably for certain types of analogical word relationships than for others, but these studies have not offered a convincing account for why this is the case. We arrive at such an account through an experiment that targets a wide variety of analogy questions and defines a baseline condition to more accurately measure the efficacy of our system. We find that the most reliably solvable analogy categories involve either 1) the application of a morpheme with clear syntactic effects, 2) male--female alternations, or 3) named entities. These broader types do not pattern cleanly along a syntactic--semantic divide. We suggest instead that their commonality is distributional, in that the difference between the distributions of two words in any given pair encompasses a relatively small number of word types. Our study offers a needed explanation for why analogy tests succeed and fail where they do and provides nuanced insight into the relationship between word distributions and the theoretical linguistic domains of syntax and semantics.
