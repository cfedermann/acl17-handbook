In this paper, we use qualitative research methods to investigate the attitudes of social media users towards the (opt-in) integration of social media data with routine mental health care and diagnosis. Our investigation was based on secondary analysis of a series of five focus groups with Twitter users, including three groups consisting of participants with a self-reported history of depression, and two groups consisting of participants without a self reported history of depression. Our results indicate that, overall, research participants were enthusiastic about the possibility of using social media (in conjunction with automated Natural Language Processing algorithms) for mood tracking under the supervision of a mental health practitioner. However, for at least some participants, there was skepticism related to how well social media represents the mental health of users, and hence its usefulness in the clinical context.
