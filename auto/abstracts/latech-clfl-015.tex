Literary genres are commonly viewed as being defined in terms of content and stylistic features. In this paper, we focus on one particular class of lexical features, namely emotion information, and investigate the hypothesis that emotion-related information correlates with particular genres. Us- ing genre classification as a testbed, we compare a model that computes lexicon- based emotion scores globally for complete stories with a model that tracks emotion arcs through stories on a subset of Project Gutenberg with five genres. Our main findings are: (a), the global emotion model is competitive with a large-vocabulary bag-of-words genre classifier (80\%F1); (b), the emotion arc model shows a lower performance (59 \% F1) but shows complementary behavior to the global model, as indicated by a very good performance of an oracle model (94 \% F1) and an improved performance of an ensemble model (84 \% F1); (c), genres differ in the extent to which stories follow the same emotional arcs, with particularly uniform behavior for anger (mystery) and fear (ad- ventures, romance, humor, science fiction).
