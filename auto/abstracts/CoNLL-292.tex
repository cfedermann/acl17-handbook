Language acquisition can be modeled as a statistical inference problem: children use sentences and sounds in their input to infer linguistic structure. However, in many cases, children learn from data whose statistical structure is distorted relative to the language they are learning.  Such distortions can arise either in the input itself, or as a result of children's immature strategies for encoding their input.  This work examines several cases in which the statistical structure of children's input differs from the language being learned.  Analyses show that these distortions of the input can be accounted for with a statistical learning framework by carefully considering the inference problems that learners solve during language acquisition
