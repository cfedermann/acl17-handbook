Human understanding of narrative is mainly driven by reasoning about causal relations between events and thus recognizing them is a key capability for computational models of language understanding. Computational work in this area has approached this via two different routes: by focusing on acquiring a knowledge base of common causal relations between events, or by attempting to understand a particular story or macro-event, along with its storyline. In this position paper, we focus on knowledge acquisition approach and claim that newswire is a relatively poor source for learning fine-grained causal relations between everyday events. We describe experiments using an unsupervised method to learn causal relations between events in the narrative genres of first-person narratives and film scene descriptions. We show that our method learns fine-grained causal relations, judged by humans as likely to be causal over 80\% of the time. We also demonstrate that the learned event pairs do not exist in publicly available event-pair datasets extracted from newswire.
