Diagnosis autocoding services and research intend to both improve the productivity of clinical coders and the accuracy of the coding. It is an important step in data analysis for funding and reimbursement, as well as health services planning and resource allocation. We investigate the applicability of deep learning at autocoding of radiology reports using International Classification of Diseases (ICD). Deep learning methods are known to require large training data. Our goal is to explore how to use these methods when the training data is sparse, skewed and relatively small, and how their effectiveness compares to conventional methods. We identify optimal parameters that could be used in setting up a convolutional neural network for autocoding with comparable results to that of conventional methods.
