Enjambment takes place when a syntactic unit is broken up across two lines of poetry, giving rise to different stylistic effects. In Spanish literary studies, there are unclear points about the types of stylistic effects that can arise, and under which linguistic conditions. To systematically gather evidence about this, we developed a system to automatically identify enjambment (and its type) in Spanish. For evaluation, we manually annotated a reference corpus covering different periods. As a scholarly corpus to apply the tool, from public HTML sources we created a diachronic corpus covering four centuries of sonnets (3750 poems), and we analyzed the occurrence of enjambment across stanzaic boundaries in different periods. Besides, we found examples that highlight limitations in current definitions of enjambment.
