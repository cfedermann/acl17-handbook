Interest in neural machine translation has grown rapidly as its effectiveness has been demonstrated across language and data scenarios.  New research regularly introduces architectural and algorithmic improvements that lead to significant gains over ``vanilla'' NMT implementations.  However, these new techniques are rarely evaluated in the context of previously published techniques, specifically those that are widely used in state-of-the-art production and shared-task systems.  As a result, it is often difficult to determine whether improvements from research will carry over to systems deployed for real-world use.  In this work, we recommend three specific methods that are relatively easy to implement and result in much stronger experimental systems.  Beyond reporting significantly higher BLEU scores, we conduct an in-depth analysis of where improvements originate and what inherent weaknesses of basic NMT models are being addressed.  We then compare the relative gains afforded by several other techniques proposed in the literature when starting with vanilla systems versus our stronger baselines, showing that experimental conclusions may change depending on the baseline chosen.  This indicates that choosing a strong baseline is crucial for reporting reliable experimental results.
