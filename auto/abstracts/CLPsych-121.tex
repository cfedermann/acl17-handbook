Individuals on social media may reveal themselves to be in various states of crisis (e.g. suicide, self-harm, abuse, or eating disorders). Detecting crisis from social media text automatically and accurately can have profound consequences. However, detecting a general state of crisis without explaining why has limited applications. An explanation in this context is a coherent, concise subset of the text that rationalizes the crisis detection. We explore several methods to detect and explain crisis using a combination of neural and non-neural techniques. We evaluate these techniques on a unique data set obtained from Koko, an anonymous emotional support network available through various messaging applications. We annotate a small subset of the samples labeled with crisis with corresponding explanations. Our best technique significantly outperforms the baseline for detection and explanation.
