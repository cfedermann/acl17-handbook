Probabilistic topic models are important tools for indexing, summarizing, and analyzing large document collections by their themes.  However, promoting end-user understanding of topics remains an open research problem.  We compare labels generated by users given four topic visualization techniques—word lists, word lists with bars, word clouds, and network graphs—against each other and against automatically generated labels.  Our basis of comparison is participant ratings of how well labels describe documents from the topic. Our study has two phases{:} a labeling phase where participants label visualized topics and a validation phase where different participants select which labels best describe the topics' documents.  Although all visualizations produce similar quality labels, simple visualizations such as word lists allow participants to quickly understand topics, while complex visualizations take longer but expose multi-word expressions that simpler visualizations obscure.  Automatic labels lag behind user-created labels, but our dataset of manually labeled topics highlights linguistic patterns (e.g., hypernyms, phrases) that can be used to improve automatic topic labeling algorithms.