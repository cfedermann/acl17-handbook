\begin{bio}
  {\bfseries Jennifer Wortman Vaughan} is a researcher at Microsoft Research,
  New York City. She is interested in developing general methods that allow us
  to reason formally about the performance of algorithms with human components
  in the same way that traditional computer science techniques allow us to
  formally reason about algorithms that run on machines alone.

\end{bio}

\begin{tutorial}
  {Making Better Use of the Crowd}
  {tutorial-final-006}
  {\daydateyear, \tutorialafternoontime}
  {\TutLocF}

Over the last decade, crowdsourcing has been used to harness the power of human
computation to solve tasks that are notoriously difficult to solve with
computers alone, such as determining whether or not an image contains a tree,
rating the relevance of a website, or verifying the phone number of a business.
The natural language processing community was early to embrace crowdsourcing as
a tool for quickly and inexpensively obtaining annotated data to train NLP
systems. Once this data is collected, it can be handed off to algorithms that
learn to perform basic NLP tasks such as translation or parsing. Usually this
handoff is where interaction with the crowd ends. The crowd provides the data,
but the ultimate goal is to eventually take humans out of the loop. Are there
better ways to make use of the crowd?

In this tutorial, I will begin with a showcase of innovative uses of
crowdsourcing that go beyond data collection and annotation. I will discuss
applications to natural language processing and machine learning, hybrid
intelligence or “human in the loop” AI systems that leverage the complementary
strengths of humans and machines in order to achieve more than either could
achieve alone, and large scale studies of human behavior online. I will then
spend the majority of the tutorial diving into recent research aimed at
understanding who crowdworkers are, how they behave, and what this should
teach us about best practices for interacting with the crowd.

\end{tutorial}